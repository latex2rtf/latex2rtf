\documentclass[12pt]{article}
\usepackage{multicol,french,smallcap,a4huge,narrowit,smaller,moresize,textcomp,musixtex,amsmath}
\usepackage[cp850]{inputenc}
\begin{document}
 \title{\rm\sl\bfseries \{Candidature\}\footnote{Texte d'essai.} de Daniel 
Taupin\\au {\scfamily Comit�} \upshape Directeur \mdseries national {\scfamily
Ffme}:\\<<~\slshape faire de la {\scfamily Ffme}{} un vrai service public~>>}
 \author {Le m�me\footnote{Pour ne pas se r�p�ter}\}} \maketitle

\rm\upshape\mdseries

\section*{Section non num�rot�e}

 Malgr\'e $1_2$ et  $1_\pi$  et  ${{1_\pi}}$ et $1_{\pi}$ de nets \& progr\`es comme le nouveau si\`ege \`a Paris, tout
n'est pourtant pas parfait dans la {\scfamily Ffme}{} nationale~: elle
fonctionne actuellement $\sqrt{x+1}$ plus comme une {\bf\slshape administration} qui g\`ere
les assurances, cotisations, finances nationales\footnote{Finances de la
{\scfamily Ffme}
Nationale, pas de l'\'Etat~!}, stages de formation et I\textsuperscript{II}
surtout comp\'etitions, que comme --une {\bf\slshape service public} pour les
licenci\'es ordinaires. 1\ier{} {\Large 1\ier} {\huge 1\ier}  {\HUGE 1\ier{} 23\textdegree}Pour qu'elle le devienne, j'aurai \`a c\oe ur de
[faire] am\'eliorer trois domaines (avec un n-accent~:`\'n', et en majuscule
``\'N'', le w-circonflexe `\^w' et `\^W', \`W, \"z, ):

Avec des c�dilles: s-c�dille \c{s}, \c{S}, \c{t}, \c{T}, \c{a}, \c{A}.

Avec des ronds: s-rond \r{s}, \r{S}, \r{t}, \r{T}, \r{a}, \r{A}.

Avec des haceks: s-hacek \v{s}, \v{S}, \v{c}, \v{C}, \v{t}, \v{T}, \v{a}, \v{A}.

 Avec des Under-Dots: s-dot \d{s},
  \d{S},
  c-dot \d{c}, \d{C}, \d{t}, \d{T}, \d{a}, \d{A}.

 Avec des Macrons: s-macron \={s},
  \={S},
  c-macron \={c}, \={C}, \={t}, \={T}, \={a}, \={A}.

Ici le logo FFME={\unitlength=1mm
\begin{picture}(63.67,54.44)
\put(0,54.44){\special{em:graph c:/taupin/ffm/logos/ffmlogof.pcx}}
\end{picture}}=et le texte \`a la suite.



 From the \LaTeX-er's view point, output (text and moreover maths) is much
better than the results obtained by average `Wordists', especially with respect
to lists. Fox example the famous formula using the discriminant
$\sqrt{b^2-4ac}$ to provide the solution $
x = {\frac{-b\pm\sqrt{b^2-4ac}}{2a}}
$
of the second degree equations:\begin{displaymath}
x = {{-b\pm\sqrt{b^2-4ac}}\over{2a}}
\end{displaymath}

Autre chose:$$
x = {{-b\pm\sqrt{b^2-4ac}}\over{2a}}
$$


\begin{displaymath}
x = \int_{-\infty}^{+2\pi}{{-b\pm\sqrt{b^2-4\mathbf{ac}}}\over{2a}}\textbf{d}x
\end{displaymath}


\begin{verbatim}
 \begin{equation}
 z = x^3ww * y^{2q}z
 \end{equation}
\end{verbatim} 
 Ici, avec plain display.

 $$
 z = x^4_Ww * y^{2q}z
 $$

\par
Ici, avec /begin\{displaymath\}

 \begin{displaymath}
 z = x^5ww * y^{2q}z
 \end{displaymath}


\par
Ici, avec /begin\{math\}

L'�quation est dans la ligne \begin{math}
 z = x^6ww * y^{2q}z
 \end{math} et ici la ligne se termine.


 \par
Ici, avec /crochets/\[
 z = x^7ww * y^{2q}z
 \]

 Ici, je fais r�f�rence au <<~texte � itemizer par Word~>>, cf.
\S\ref{Itmword} � la p. \pageref{Itmword}.

 Et maintenant de la musique:\begin{music}
\parindent10mm
\setname1{Piano}
\setstaffs12
\generalmeter{\meterfrac44}
\startextract
\Notes\ibu0f0\qb0{cge}\tbu0\qb0g|\hl j\en
\Notes\ibu0f0\qb0{cge}\tbu0\qb0g|\ql l\sk\ql n\en
\bar
\Notes\ibu0f0\qb0{dgf}|\qlp i\en
\notes\tbu0\qb0g|\ibbl1j3\qb1j\tbl1\qb1k\en
\Notes\ibu0f0\qb0{cge}\tbu0\qb0g|\hl j\en
\endextract
\end{music}-et la fin de la musique.

\section*{Autre section non num\'erot\'e}
 Malgr\'e de nets progr\`es comme le nouveau si\`ege \`a Paris, tout n'est
pourtant pas parfait dans la {\scfamily Ffme}{} nationale~: elle fonctionne
actuellement plus comme une {\bf\slshape administration} qui g\`ere les
assurances, cotisations, finances nationales\footnote{Finances de la
{\scfamily Ffme} Nationale, pas de l'\'Etat~!}, stages de formation et surtout
comp\'etitions, que comme une {\bf\slshape service public} pour les
licenci\'es ordinaires. Pour qu'elle le devienne, j'aurai\cite{guidepc} et
\cite{amenage} (ou bien \cite{amenagex}) \`a c\oe ur de [faire] am\'eliorer trois domaines~: (ici
\'equations)

 $a b c = {{\sqrt{5}}\over{1_{\pi}+1_\pi}} $

 $$z = x^6ww * y^{2q}z $$


\begin{enumerate}
 \item Z\'eroi�me item...
 \item\OE uvre de {\Large communication\footnote{Ceci pour utiliser la lettre \OE.}
f\'ed\'erale}
\end{enumerate}

Encore...

\begin{enumerate}
 \item Z\'eroi�me item...
 \item\OE uvre de {\Large communication\footnote{Ceci pour utiliser la lettre \OE.}
f\'ed\'erale}
 \item L'image de marque \relax CO$\null^\textbf{2}$ du CO$_\textbf{2 multipli� par
x}$ de la {\scfamily Ffme}, $\sqrt{2}$encore L'image de marque de la {\scfamily Ffme},
encore L'image de marque de la {\scfamily Ffme}, encore L'image de marque de
la {\scfamily Ffme}, encore .

 \par Un paragraphe isol\'e dans l'\'enum\'eration~: est-il align� avec lle reste,
d�but de la ligne et suite du paragraphe tout entier qui peut �tre tr�s long.
 \item Se donner les moyens d'une politique de service public.

 \par Un paragraphe isol\'e dans l'\'enum\'eration~: est-il align� avec lle reste,
d�but de la ligne et suite du paragraphe tout entier qui peut �tre tr�s long.
\end{enumerate}

\section{La communication f\'ed\'erale}

\bigskip\parindent 2cm

Ici parindent 2cm. Nous avons bien une responsable de la communication, mais
le mot <<~communication~>> est ambigu~: il ne s'agit pas seulement de
communiquer avec d'\'eventuels partenaires, publicistes ou sponsors, mais de
communiquer avec le licenci\'e ordinaire, celui qui attend de la
F\'ed\'eration des services, des aides ou des informations. Cela n\'ecessite~:
\section{La communication f\'ed\'erale}

\parindent 1cm

{\Huge $$a b c = {{\sqrt{5}}\over{\pi}} $$

 $$z = x^7ww * y^{2q}z $$}

Ici parindent 1cm apr\'es deux Huge \'equations. Nous avons bien une responsable de la communication, mais le mot
<<~communication~>> est ambigu~: il ne s'agit pas seulement de communiquer
avec d'\'eventuels partenaires, publicistes ou sponsors, mais de communiquer
avec le licenci\'e ordinaire, celui qui attend de la F\'ed\'eration des
services, des aides ou des informations. Cela n\'ecessite~: \begin{enumerate}
 \item que la {\scfamily Ffme}{} soit un v\'eritable un office de
renseignements (techniques, administratifs, juridiques --- article 10, alin\'ea
7 des statuts) pour les licenci\'es,
\item qu'elle diffuse aux licenci\'es les informations {\bf\slshape qui les
int\'eressent},
\item qu'elle soit efficace dans l'\'edition d'ouvrages techniques de
r\'ef\'erence,
\item que la communication entre licenci\'es et direction/pr\'esidence soit
existante, rapide, efficace,
\item qu'elle soit efficace dans l'\'edition d'ouvrages techniques de
r\'ef\'erence,
\item que la communication entre licenci\'es et direction/pr\'esidence soit
existante, rapide, efficace,
\item qu'elle soit efficace dans l'\'edition d'ouvrages techniques de
r\'ef\'erence,
\item que la communication entre licenci\'es et direction/pr\'esidence soit
existante, rapide, efficace,
\item qu'elle soit efficace dans l'\'edition d'ouvrages techniques de
r\'ef\'erence,
\item que la communication entre licenci\'es et direction/pr\'esidence soit
existante, rapide, efficace,
\item qu'elle soit efficace dans l'\'edition d'ouvrages techniques de
r\'ef\'erence,
\item que la communication entre licenci\'es et direction/pr\'esidence soit
existante, rapide, efficace,
\item qu'elle soit efficace dans l'\'edition d'ouvrages techniques de
r\'ef\'erence,
\item que la communication entre licenci\'es et direction/pr\'esidence soit
existante, rapide, efficace,
\item qu'elle soit efficace dans l'\'edition d'ouvrages techniques de
r\'ef\'erence,
\begin{enumerate}
 \item que la communication entre licenci\'es et direction/pr\'esidence soit
existante, rapide, efficace,
\begin{enumerate}
 \item qu'elle soit efficace dans l'\'edition d'ouvrages techniques de
r\'ef\'erence,
\item que la communication entre licenci\'es et direction/pr\'esidence soit
existante, rapide, efficace,
\begin{enumerate}
 \item qu'elle soit efficace dans l'\'edition d'ouvrages techniques de
r\'ef\'erence,
\item que la communication entre licenci\'es et direction/pr\'esidence soit
existante, rapide, efficace,
\item qu'elle soit efficace dans l'\'edition d'ouvrages techniques de
r\'ef\'erence,
\end{enumerate}
 \item qu'elle soit efficace dans l'\'edition d'ouvrages techniques de
r\'ef\'erence,
\end{enumerate}
\end{enumerate}
 \item que la communication entre licenci\'es et direction/pr\'esidence soit
existante, rapide, efficace,
\item qu'elle soit efficace dans l'\'edition d'ouvrages techniques de
r\'ef\'erence,
\item que la communication entre licenci\'es et direction/pr\'esidence soit
existante, rapide, efficace,
\item qu'elle soit efficace dans l'\'edition d'ouvrages techniques de
r\'ef\'erence,
\item que la communication entre licenci\'es et direction/pr\'esidence soit
existante, rapide, efficace,
\end{enumerate}

\subsection{La {\scfamily Ffme}{}, un office de renseignements pour les
licenci\'es}
\begin{itemize}
 \item Ne jamais laisser sans r\'eponse un courrier de demande de
renseignements, quitte \`a fabriquer des lettres de r\'eponses standard \`a
toutes les questions pos\'ees fr\'equemment,
\par
Et un paragraphe sans <<~bullet~>> au d\'ebut, ceci pour savoir si Word en
rajoute ou n'en met que quand on le lui demande...

\item avoir en permanence au si\`ege une personne capable de r\'epondre \`a
toute question classique et qui sache rapidement trouver les r\'eponses aux
questions plus difficiles,
\item diffuser ces informations non seulement par courrier ou r\'eponse au
t\'el\'ephone, mais par tous les moyens actuels (minitel, serveurs Internet,
etc.).
\begin{itemize}
 \item bla bla bla.
\item blah blah blah, blah blah blah, blah blah blah, blah blah blah, blah blah
blah, blah blah blah, blah blah blah, blah blah blah, blah blah blax, blax blax
blax, blax blax blax, blax blax blax, blax blax blax, blah blah blah, blah blah
blah, blah blah blah, blah blah blah, blah blah blah, blah blah blah, blah blah
blah, blah blah blah, blah blah blah, blah blah blah, blah blah blah, blah blah
blah, blah blah blah,
\item et la fin du bla bla bla interminable...
\end{itemize}
 \item diffuser ces informations non seulement par courrier ou r\'eponse au
t\'el\'ephone, mais par tous les moyens actuels (minitel, serveurs Internet,
etc.).
\end{itemize}

\subsection{Texte \'a itemizer par Word\label{Itmword}}

Ne jamais laisser sans r\'eponse un courrier de demande de renseignements,
quitte \`a fabriquer des lettres de r\'eponses standard \`a toutes les
questions pos\'ees fr\'equemment,

Avoir en permanence au si\`ege une personne capable de r\'epondre \`a toute
question classique et qui sache rapidement trouver les r\'eponses aux questions
plus difficiles,

Diffuser ces informations non seulement par courrier ou r\'eponse au
t\'el\'ephone, mais par tous les moyens actuels (minitel, serveurs Internet,
etc.).

Fin itemization par word.

 \begin{multicols}2
\subsection{Diffuser aupr\`es des licenci\'es les informations {\bf\slshape qui
les int\'eressent}} Cela signifie un bulletin p\'eriodique d'information aux
licenci\'es, ou au moins aux pr\'esidences d'associations, bulletin qui ne soit
pas un prospectus d'autosatisfaction mais qui r\'eponde par avance aux
questions que se posent les licenci\'es~: probl\`emes d'assurance, probl\`emes
de responsabilit\'e, quels sites de pratique pr\'ef\'erer selon son niveau,
quel mat\'eriel utiliser en alpinisme, en escalade, en randonn\'ee, en
\'equipement de sites, en exp\'edition, etc.

\subsection{Entreprendre \label{Entreprendre}efficacement l'\'edition
d'ouvrages techniques de r\'ef\'erence}

\begin{itemize}
 \item Am\'elioration technique (donc informatique) des m\'ethodes de saisie
des ouvrages (r\"apidit\'e de mise \`a jour, c\"ontr\^{o}le rig\"oureux du
cont\"enu et de la mise en page, ma\"is, ma\"\i{}zena, ma\^{\i}zena, b\"ureau, etc.),
\item ce qui entra\^\i{}ne le besoin d'avoir au si\`ege (ou pas loin) des
personnes capables de saisir, de relire et de mettre en page ces ouvrages sans
la lourdeur de l'appel \`a une maison d'\'edition.
\end{itemize}
 \subsection{Une communication existante, rapide et efficace entre licenci\'es,
comit\'es territoriaux et direction/pr\'esidence}
\begin{itemize}
 \item Savoir utiliser les moyens modernes de communication (courrier
\'electronique, Internet, serveur minitel) pour informer rapidement les
comit\'es, {\bf\slshape et inversement}.
\item Utiliser ces moyens pour permettre aux licenci\'es --- au moins ceux qui
savent les utiliser ou qui ont des amis qui savent les utiliser --- de
conna\^\i{}tre {\bf\slshape en temps r\'eel} (c'est-\`a-dire en quelques jours
et non en quelques mois comme la publication dans la presse) les annonces de
stages, de comp\'etitions, leurs modalit\'es d'inscription, leurs suppressions,
leur saturation \'eventuelle, les prises de position politique de la {\scfamily
Ffme}, les nouveaut\'es techniques (notamment pour la s\'ecurit\'e).
\end{itemize}
 \section{Une autre image de marque de la {\scfamily Ffme}} Je ne critique pas
le temps pass\'e \`a travailler pour les comp\'etitions et le haut niveau, mais
d'une part le temps pass\'e \`a en discuter dans les instances f\'ed\'erales,
et surtout le fait que \c Ca \c{c}occupe beaucoup trop de place dans nos
publications~: on y parle surtout de ce qui int\'eresse les dirigeants et
certains m\'edias, et pas assez de ce qui int\'eresse la majorit\'e des
licenci\'es, \`a savoir la formation, la s\'ecurit\'e et l'am\'enagement des
sites, travaux r\'ealis\'es il est vrai surtout par les comit\'es territoriaux.

\section{Se donner les moyens d'une politique de service public}

\subsection{Les moyens techniques au si\`ege f\'ed\'eral}

Les exigences de communication ne n\'ecessitent pas seulement de la bonne
volont\'e, elles n\'ecessitent aussi des comp\'etences techniques
(informatiques, r\'edactionnelles et bureautiques). Cela requiert le
recrutement sous une forme ou une autre d'un organisateur de ces moyens de
travail, c'est-\`a-dire avec de bonnes comp\'etences bureautiques,
documentalistes, informatiques, voire journalistiques. Comme pour le
comptabilit\'e, ces comp\'etences sont bien plus importantes que la
sacro-sainte appartenance \`a <<~Jeunesse et Sports~>>...

\subsection{Donner aux comit\'es territoriaux les moyens financiers}

En effet, sur le budget f\'ed\'eral consolid\'e, les comit\'es territoriaux
n'encaissent qu'environ un dixi\`eme des recettes. Or ce sont eux qui font
marcher la maison, hormis les op\'erations de prestige \`a caract\`ere
national. Pour qu'ils aient des moyens acceptables, il faudrait~:
\begin{itemize}
 \item d'une part augmenter la part des comit\'es territoriaux dans les
finances globales de la {\scfamily Ffme}{},
\item d'autre part avoir des r\`egles de r\'epartition non rigides,
c'est-\`a-dire avec un minimum calcul\'e automatiquement, mais possibilit\'e
d'attributions compl\'ementaires sur dossier justificatif.
\end{itemize}

 \end{multicols}
\subsection{Avoir un comit\'e directeur politique}

Le Comit\'e Directeur n'a pas pour fonction principale d'\'ecouter le
Pr\'esident et de r\'egler les probl\`emes financiers, il lui appartient aussi
de discuter de la politique sportive f\'ed\'erale~: quelles activit\'es
promouvoir, \'elitisme ou sport pour tous, \^etre \`a la remorque des m\'edias
ou d\'ecider de ses choix politiques, etc.

Il est clair que ceci n'est pas vraiment au choix d'un membre ordinaire du
comit\'e directeur~: c'est le bureau qui a le pouvoir de mettre les points
politiques \`a l'ordre du jour, ou au contraire de les enterrer en se
contentant de programmer les affaires courantes...

\begin{center}
 \Large{\sl\bfseries Curriculum vit\ae}
\end{center}

\begin{description} \item[\^Age :] 60 ans.
\item[Situation de famille :] mari\'e 3 enfants, 3,9 petits-enfants.
\item[Profession :] directeur de recherches au CNRS, enseignant en informatique
appliqu\'ee \`a l'Universit\'e de Paris-Sud. 
 \item[Dipl\^omes :] ing\'enieur EPCI, docteur \`es sciences.
\item[Langues :] Anglais, italien (parl\'es couramment), allemand, espagnol.
\item[Disciplines {\sl f\'ed\'erales} pratiqu\'ees :] escalade, alpinisme,
randonn\'ee (\`a pied et raquettes), exp\'editions.

\par Ici un paragraphe isol� dans la description. Ce paraphe faire plusieurs
lignes et je rajoute une bla bla bla bla pour occuper le terrain et bien voir
ce que �a fait r�ellement.
\item[Activit\'es f\'ed\'erales :] secr\'etaire comit\'e r\'egional
\^Ile-de-France, membre du Comit\'e Sportif Escalade, de la Commission
Environnement, de la Commission des Sites Naturels d'Escalade.
\item[Clubs :] GUMS Paris, CAF \^Ile-de-France.
\item[Licence 1997 :] num\'ero 7502A000050 (au GUMS).
\end{description}

Ce texte est en dehors de les description.

\begin{thebibliography}1
\bibitem{amenage}Am�nagement et �quipement d'un SNE
\bibitem{guidepc}Guide des SNE de France
\bibitem{bleau}Escalade � Blean, les Trois Pignons
\end{thebibliography}

Ici un paragraphe hors bibliographie. Il est ridiculement long pour qu'il occupe plusieurs lignes
et me permette de v�rifier la r�tablissement de l'indentation de paragraphe.


\medskip Et un autre paragraphe plus long pour qu'il occupe plusieurs lignes
et me permette de v�rifier la r�tablissement de l'indentation de paragraphe.

\begin{thebibliography}1
\bibitem{amenagex}[ASNE]Am�nagement et �quipement d'un SNE (deuxi�me)
\bibitem{guidepcx}Guide des SNE de France
\bibitem{bleaux}Escalade � Blean, les Trois Pignons
\end{thebibliography}

Ici un paragraphe hors bibliographie. Il est ridiculement long pour qu'il occupe plusieurs lignes
et me permette de v�rifier la r�tablissement de l'indentation de paragraphe.


\medskip Et un autre paragraphe plus long pour qu'il occupe plusieurs lignes
et me permette de v�rifier la r�tablissement de l'indentation de paragraphe.

\end{document}

