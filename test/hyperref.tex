\documentclass{article}
\usepackage{hyperref}
\usepackage{nameref}
\newif\iflatextortf

\begin{document}

\newcommand{\myref}[1]{`\nameref{#1}' (see p.\pageref{#1})}

\iflatextortf
\newcommand{\myurl}[1]{\nolinkurl{#1}}
\let\letequalurl=\nolinkurl
\let\leturl\nolinkurl
\else
\let\myurl\url
\let\letequalurl=\url
\let\leturl\url
\fi

\section{The first section}
hello a 
\href{http://www-fourier.ujf-grenoble.fr:80/cgi-bin/zbfr/ZB/math-fr.html?AU=Taylor,+R&format=complete&type=html&maxdocs=10}{weird}
reference

a second reference 
\href{http://www.tug.org/}{tug}
to the \TeX{} website 

which when applied as follows:

\section{Random Stuff}\label{randomlabel}
This is very dependent on \myref{randomlabel}.


\subsection{url}
We can use the \verb#\url# command to do this as
well like so: \url{http://www.tug.org}.  

The \verb#\nolinkurl{}# is similar but non-functional: \nolinkurl{http://www.tug.org}. 

Now this link \myurl{http://latex2rtf.sourceforge.net/some%20link.html} does not really
exist, but it should show up in the RTF document without a hyperlink.  It should still
have a link in the PDF file created using \texttt{pdflatex}.

\section{About paragraph breaks}

There should be a paragraph break between these two urls.

\url{http://www.tug.org/one}. 

\url{http://www.tug.org/two}. 

%\hypertarget{TestTarget}{goodbye}

\end{document}

\section{Testing the let command}

These don't work yet, that is why there is an \verb#\end{document}# above.

\iflatextortf
\let\letequalurl=\nolinkurl
\let\leturl\nolinkurl
\else
\let\letequalurl=\url
\let\leturl\url
\fi

Here is a test of the \verb#\let\letequalurl=\nolinkurl# implementation.  The following shouls be
a simple url without a link. \letequalurl{http://simple}

Here is a test of the \verb#\let\leturl\nolinkurl# implementation.  The following shouls be
a simple url without a link.  \leturl{http://just.as.simple.com}

\end{document}