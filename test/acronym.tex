\documentclass[a4paper,11pt]{article}
\usepackage[T1]{fontenc}
\usepackage[utf8x]{inputenc}

\setlength{\oddsidemargin}{0pt}
\setlength{\evensidemargin}{0pt}
\setlength{\textwidth}{6in}
\setlength{\parindent}{0pt}

%% tested packages
\usepackage[printonlyused,withpage]{acronym}
\usepackage{srcltx}
\usepackage{cite}
\usepackage{nameref} %% for more complex labels
\usepackage{ifpdf}
%% \usepackage{harvard}
%% backport in progress
%% \usepackage{graphics}
%% \graphicspath{{./graphs}{./dibujos}}
%% end tested packages
\title{acronym and srcltx support for LaTEX2rtf}
\author{Pedro A. Aranda Guti\'errez}
\begin{document}
\maketitle
\acused{bgp}
\section{Test section}
This is a test file for the acronyms. Remember we have just set:
\begin{verbatim}
\acused{bgp}
\end{verbatim}
But I will leave \ac{undef} undefined!

\subsection{Test text}
\label{sec:test}

Each time a \ac{bgp} advertisement traverses an \ac{as}, the \ac{aspath}
attribute is modified. Additionally, many \acp{as} introduce additional
changes in the \ac{aspath} for traffic engineering purposes.

Forget about using \acfi{RIP} in an interdomain environment. This
should have expanded the \texttt{RIP} acronym, without marking it as
used. The next time it is used, it will appear completely expanded.

But it is not only communications which have strange acronyms. Look at
politics: You have one \ac{MP} and there is no problem. Even when you
don't want to waste too much time, you are on the safe side: \acp{MP}.
But when you bother about all them, you get in trouble with \aclp{MP}.

\section{Acronym definition macros}

Using following acronym macros

\begin{verbatim}
\usepackage[printonlyused,withpage]{acronym}

\begin{acronym} [IS-IS]
\acrodef{aspath}[AS\_PATH]{Autonomous System Path}
\acrodef{as}[AS]{Autonomous System}
\acrodefplural{as}[ASes]{Autonomous Systems}
\acrodef{MP}{Member of Parliament}
\acrodefplural{MP}{Members of Parliament}
\acro{bgp}[BGP-4]{Border Gateway Protocol}
\acro{isis}[IS-IS]{IS-IS Protocol}
\acro{RIP}{RIPv2 Routing Protocol}
\acro{NA}[NA]{Number of Avogadro \acroextra{ (See Section~\ref{sec:test})}}
\end{acronym}
\end{verbatim}

Only \textbf{\ac{bgp}} and \textbf{\ac{RIP}} should appear as an
acronym in the acronym section. This line should have the acronym
\ac{RIP} expanded. 
\begin{verbatim}
\acfi{RIP}
\end{verbatim}
doesn't mark the acronym as used! (see generated acronym.dvi)

\section{Shortcuts}
\label{sec:shortcut}

Language options can be included in the document class:
\begin{verbatim}
\documentclass[a4paper,11pt,spanish]{article}
\end{verbatim}
to print the correct Acronym section header. 

Needs \texttt{ACRONAME} to be correctly configured in the latex2rtf
language specific configuration file. The main and only difference with
the original package is that the \textbf{Acronyms} section header will
always be generated.

The text for the \ac{NA}, which is defined as
\begin{verbatim}
\acro{NA}[NA]{Number of Avogadro \acroextra{ (See Section~\ref{sec:test})}}
\end{verbatim}
where \texttt{ref\{sec:test\}} refers to Section~\ref{sec:test} in
Page~\pageref{sec:test}.

It is also necessary to scan the .aux file for other things, like
citations.. Let's see if I have gotten it right and I can cite the
multivendor interworking book \cite{Goralski-cisco-juniper} or the
Defcon paper \cite{defcon16-bgp}

\section*{Acronyms}
\begin{acronym}[BGP-4]
\acrodef{aspath}[AS\_PATH]{Autonomous System Path}
\acrodef{as}[AS]{Autonomous System}
\acrodefplural{as}[ASes]{Autonomous Systems}
\acrodef{MP}{Member of Parliament}
\acrodefplural{MP}{Members of Parliament}
\acro{bgp}[BGP-4]{Border Gateway Protocol}
\acro{isis}[IS-IS]{IS-IS Protocol}
\acro{RIP}{RIPv2 Routing Protocol}
\acro{NA}[NA]{Number of Avogadro \acroextra{ (See Section~\ref{sec:test})}}
\end{acronym}

\bibliographystyle{abbrv}
\bibliography{acronym}

\end{document}
