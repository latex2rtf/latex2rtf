\documentclass[11pt]{report} % ergaenze `twoside', wenn gewuenscht!
\usepackage{ngerman}

\title{Test-Dokument f"ur "`german.sty"'}
\author{Wilfried Hennings}
\date{2001-09-10}

\begin{document}
\maketitle

\tableofcontents

\chapter{Einleitung}
Dies ist ein nicht besonders intelligenter Text zum Testen von
Dokumenten, die die Besonderheiten des Pakets "`german.sty"' verwenden.

Das Paket muss mit folgender Syntax in der Preambel des Dokuments
eingebunden werden:\\
\verb|\usepackage{german}|\\
f"ur die "`alte"' Rechtschreibung oder \\
\verb|\usepackage{ngerman}|\\
f"ur die "`neue"' Rechtschreibung 


\chapter{Die Besonderheiten deutscher Texte}
\section{Umlaute}

Wichtig in deutschen Texten sind vor allem die Umlaute.

\subsection{Umlaute ohne das Paket "`german.sty"'}
Ohne das Paket "`german.sty"' m"ussen Umlaute so kodiert werden, wie in
Tabelle~\ref{T1} auf Seite \pageref{T1} dargestellt.
\begin{table}[ht]
\"A = \verb|\"A|\\
\"O = \verb|\"O|\\
\"U = \verb|\"U|\\
\"a = \verb|\"a|\\
\"o = \verb|\"o|\\
\"u = \verb|\"u|\\
\ss{} = \verb|\ss{}|
\caption{Tabelle der Umlaute in Standard-LaTeX} 
\label{T1}
\end{table}

\subsection{Umlaute mit dem Paket "`german.sty"'}
Mit dem Paket "`german.sty"' k"onnen Umlaute auch so kodiert werden, wie
in Tabelle~\ref{T2} auf Seite \pageref{T2} dargestellt.
\begin{table}[ht]
"A = \verb|"A|\\
"O = \verb|"O|\\
"U = \verb|"U|\\
"a = \verb|"a|\\
"o = \verb|"o|\\
"u = \verb|"u|\\
"s = \verb|"s|\\
\caption{Tabelle der Umlaute mit dem Paket "`german.sty"'} 
\label{T2}
\end{table}

\section{Beschriftungen}

Eine weitere Anpassung ist f"ur Beschriftungen der Elemente eines
Dokuments erforderlich. 

Zum Beispiel soll aus dem englischen "`page"' das deutsche "`Seite"'
werden, aus dem englischen "`bibliography"' das deutsche "`Literatur"'.
Um dies zu testen, sind in diesem Dokument verschiedene "Uberschriften,
eine Abbildung (Abbildung~\ref{A1} auf Seite \pageref{A1}), zwei
Tabellen (Tabelle~\ref{T1} auf Seite \pageref{T1} und Tabelle~\ref{T2}
auf Seite \pageref{T2}) sowie eine Bibliographie eingef"ugt. Einige
Beispiele sind der "`LaTeX2e-Kurzbeschreibung"' \cite{l2kurz} entnommen.


\begin{figure}[tb]
\vspace{4cm}
\caption{Landschaft im Nebel} 
\label{A1}
\end{figure}

\section{Anf"uhrungszeichen}

Schlie"slich sehen in deutschen Texten die Anf"uhrungszeichen anders
aus. W"ahrend im Englischen die doppelten Anf"uhrungszeichen ``so''
aussehen, sollen sie im Deutschen "`so"' aussehen. 


\begin{thebibliography}{9} 
 
\bibitem{l2kurz}
Walter Schmidt, J"org Knappen, Hubert Partl, Irene Hyna:\\
\textit{LaTeX2e-Kurzbeschreibung}


\end{thebibliography}
 
\end{document}
