\documentclass[12pt,titlepage]{article}             %Esser's rules: 10-12pt
\title{Article Review of:\\Big Brother and the Bookie}
\author{Justin Gombos}
\date{11/4/2002}
\begin{document}

  \maketitle
  \baselineskip=19pt

  \section{Societal impact of the Scarfo case}
  
  The government has once again over-stepped their power.  Fourth
  Amendment rights are conveniently tossed out the door as usual, and
  George Anastasia covers it in his article \textit{Big Brother and
    the Bookie}.  He does a decent job of informing the public that
  there is a real concern for privacy issues today.  He illustrates a
  current threat to the privacy of citizens, and his article is
  effective.
  
  The main weakness is oversight of details.  Emphasis in this paper
  will be on the lacking details.

  \section{Anastasia claims the feds bugged Scarfo's phone}
  
  The biggest flaw in Anastasia's article comes from his statement
  that the FBI did not have a wiretap warrant, yet they recorded
  conversations at the beginning of the investigation in 1999, the
  same time they seized his computer data and implanted the keyboard
  logger system (KLS).  The greatest weakness in the prosecutions case
  stemmed from not having a warrant permitting the FBI to wiretap
  Scarfo.  Yet Anastasia fails to explain how the FBI justified
  recording phone conversations of Scarfo \cite{ANASTASIA}.
  
  An explanation is important here.  While the FBI can always argue a
  strong point that keyboard monitoring is not covered in the
  wiretapping regulations, phone conversation monitoring is covered in
  great detail.  If the FBI illegally monitored Scarfo's phone line,
  then the defense can suppress all evidence collected in connection
  to that illegal wiretapping activity.  This also serves as an
  indicator that the FBI was clearly conducting an illegitimate
  investigation, thus supporting the defenses motion to suppress
  evidence connected with monitoring Scarfo's keyboard.

  \section{Technical details on the KLS overlooked}
  
  Anastasia simply describes the KLS as a sugar cube sized device that
  is planted in the keyboard and transmits keystrokes to a remote
  listening station.  An explanation like this will suffice for a
  typical non-technical reader, however, it's insufficient for
  critical analysis.
  
  With Anastasia's explanation, the reader concludes that the feds
  were collecting every keystroke.  This includes keystrokes
  transmitted off the premises.  This leads the reader to conclude
  without doubt that the target was wire-tapped.

  There are a couple reasons why Anastasia would mislead the reader in
  this way.  It could be to generate hype, or simply out of
  carelessness.  Maybe this was done to keep the article simple and
  non-technical.  
  
  The KLS actually turns out to be multiple components concealed in
  different places, such as the keyboard and within the computer case.
  This is an important distinction.  According to an affidavit
  submitted by Randall Murch, one of the components monitors the
  communications ports and is also in communication with the keyboard
  implant \cite{RANDALL}.  When a key is pressed, the keyboard device
  queries the port monitor as to whether the character made it through
  the port.  If the answer is no, the character is not considered a
  transmitted character and is recorded.  Otherwise, it's ignored.
  This is the basis for which the government denies the claim to be
  intercepting transmissions.

  \section{Sneaky FBI tricks}
  
  The FBI plotted specifically to circumvent wiretap restrictions in
  the creation of the KLS.  This motive is absent in Anastasia's
  report.  The key logger was designed such that the only messages
  exempt from eavesdropping are those sent in real-time character for
  character.  Effectively, the FBI has claimed that only instant
  messaging oriented transmissions are protected from the KLS by the
  Title III wiretapping regulations.  Email is vulnerable to
  eavesdropping, and no wiretap warrant is necessary if the
  interception occurs on the data before or after the data is
  transmitted.

  In the prosecutions rebuttal to the motion to suppress evidence,
  they stated that they only need a search warrant to obtain documents
  from a particular location.  If such documents are faxed before or
  after interception, it's immaterial and no wiretap warrant is
  necessary.  
  
  The prosecution concluded without sound reasoning that they could
  apply this same philosophy to email.  If an email intercepted before
  or after transmission, it does not fall under the Title III
  wiretapping regulation \cite{CLEARY}.  It's an absolute outrage that
  the judge accepted this reasoning.  Almost equally appalling is
  Anastasia's failure to bring this to light.
  
  The FBI made an indirect assertion that because there is some lag
  time between data entry and data transmission, eavesdropping does not
  constitute wiretapping \cite{CLEARY}.  The prosecutions case hinged
  significantly on this silly technicality, and this point should be
  one of the major points discussed in any report regarding the Scarfo
  case.

  \section{Setting the standard}

  Using technology, the feds have conceived a method for circumventing
  their responsibility in obtaining proper warrants for wiretaps.
  They can now intercept email messages without the burden of seeking
  wiretapping approval and without regard to the Fourth Amendment.  

  \section{General warrant}

  When the defense motioned for suppression of evidence, another
  strong point relates to general warrants.  Investigators cannot
  simply rummage through someone's property looking for something that
  can be used against them.  A search must be focused with an
  expectation of finding a specific piece of evidence.  Anything else
  discovered must be ignored.  ``A general warrant is described as one
  that authorizes 'a general, exploratory rummaging in a person's
  belongings,''' \cite{GELMAN}.
  
  Cleary argued that investigators recorded a minimal number of
  keystrokes to obtain Scarfo's password \cite[p.20]{CLEARY}.  He
  sited cases where it has been generally accepted to look through
  items that are not part of the search if necessary to uncover an
  item that is targeted in a search.  For example, if they are
  searching for a document that is at the bottom of a drawer, it's
  acceptable to look at other documents in the drawer long enough to
  determine if they are the document in question.

  Both the defense and Anastasia failed to mention that the
  investigators had alternative options available for capturing the
  password without capturing unnecessary information.  The KLS
  captured all keystrokes.  The FBI was aware that it was a PGP
  pass-phrase that needed to be captured.  They could have easily
  installed a software or hardware device that detects a PGP
  pass-phrase prompt, and only record the keystrokes that followed
  before enter was pressed.  
  
  Instead, the FBI captured everything, and because they captured more
  information than necessary, the defense could have made a case
  against them.  The defense did not address the prosecutions
  statement disclaiming the existence of a general warrant, and
  weakened their case.

  \section{Anastasia's article--- adequate for casual non-technical readers only}
  
  George Anastasia's article \textit{Big Brother and the Bookie}
  provides a decent summary of the U.S. governments case against
  Nicodemo Scarfo, but overlooks some important key points in the
  interest of entertaining the casual reader.  He supplies irrelevant
  details such as Scarfo's illegal video poker machines, details about
  his father and other mobster activity, and discusses how Scarfo
  dealt with rodents in his house.
  
  If Anastasia's goal was ``infotainment,'' he achieved it.  If his
  goal was to make the general public aware of the governments
  Orwellian domestic spying activities, he achieved it.  However,
  critical readers were left at a loss for details about how the
  government got away with violating Scarfo's Fourth Amendment rights.
  As a Fourth Amendment advocate, I became tired of hearing Anastasia
  go on about what happened, with little explanation as to how it
  happened.  Reading the court documents was necessary in filling in
  the blanks as to how the government got away with disposing of
  Scarfo's Fourth Amendment right.
  
  While misuse of the latest FBI gadget cost Scarfo 33 months of
  incarceration, we all pay the price of reduced civil liberties.

\begin{thebibliography}{}

\bibitem[Anastasia, 2002]{ANASTASIA}
Anastasia, G. (2002).
\newblock Big brother and the bookie: how the feds turned top-secret spy
  technology against the son of a mafia don--and made a low-level wiseguy into
  a poster boy for the fourth amendment.
\newblock {\em Mother Jones}.
\newblock http://www.motherjones.com/magazine/JF02/mafia.html.

\bibitem[Cleary, 2001]{CLEARY}
Cleary, R.~J. (2001).
\newblock Brief of the united states in opposition to defendant scarfo's
  pretrial motions.
\newblock http://www2.epic.org/crypto/scarfo/gov\_brief.pdf.
\newblock Retrieved November 3, 2002.

\bibitem[Gelman and Scoca, 2001]{GELMAN}
Gelman, N.~E. and Scoca, V.~C. (2001).
\newblock Defense motion to suppress evidence seized by the government through
  the use of a keystroke recorder.
\newblock http://www2.epic.org/crypto/scarfo/def\_supp\_mot.pdf.
\newblock Retrieved November 3, 2002.

\bibitem[Murch, 2001]{RANDALL}
Murch, R.~S. (2001).
\newblock Affidavit of randall s. murch.
\newblock http://www2.epic.org/crypto/scarfo/murch\_aff.pdf.
\newblock Received on November 3, 2002.

\end{thebibliography}

\end{document}
