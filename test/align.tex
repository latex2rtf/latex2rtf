\documentclass{article}
\begin{document}

Here we test the center environment.  The \verb#\begin{center}...\end{center}#
sequence should allow the sans serif font to leak out into the text that follows
the centered environment. 
\begin{center}
\sf\bfseries Some \verb#\sf\bfseries# centered text --- voil\'a
\end{center}
This should now be in plain roman font with the paragraph justified.
We need a lot of text to test the justification.  A short sentence would
not really be enough.  A couple of short sentences should do the trick.
I wonder what \verb#\centerline# will do in a sentence.  Here goes
the test \centerline{\it some centerline text} some more text after that
exciting test.
\begin{flushleft}
\sc Small caps flush left on line 1 \\
Small caps flush left on line 2 \\
.\\
.\\
.\\
\end{flushleft}
Now we only have the \verb#\begin{flushright}...\end{flushright}# environment
to deal with.  This paragraph should be justified on both margins. The next should
be flush with the right margin.

\begin{flushright}
\itshape 
Italics text flush right on line 1 \\
Italics text flush right on line 2 \\
.\\
.\\
.\\
\end{flushright}

More roman text at the end. More roman text at the end. More roman text at the end. 
More roman text at the end. More roman text at the end. More roman text at the end.
More roman text at the end. More roman text at the end. More roman text at the end.
More roman text at the end. More roman text at the end. More roman text at the end.
\end{document}
